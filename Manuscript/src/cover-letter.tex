\documentclass[11pt,a4paper,roman]{moderncv}        % possible options include font size ('10pt', '11pt' and '12pt'), paper size ('a4paper', 'letterpaper', 'a5paper', 'legalpaper', 'executivepaper' and 'landscape') and font family ('sans' and 'roman')
% \usepackage[spanish,es-lcroman]{babel}

% moderncv themes
\moderncvstyle{classic}                            % style options are 'casual' (default), 'classic', 'oldstyle' and 'banking'
\moderncvcolor{black}                              % color options 'blue' (default), 'orange', 'green', 'red', 'purple', 'grey' and 'black'
% \renewcommand{\familydefault}{\sfdefault}         % to set the default font; use '\sfdefault' for the default sans serif font, '\rmdefault' for the default roman one, or any tex font name
\nopagenumbers{}                                  % uncomment to suppress automatic page numbering for CVs longer than one page

% character encoding
\usepackage[utf8]{inputenc}                       % if you are not using xelatex ou lualatex, replace by the encoding you are using

\renewcommand\labelitemi{•}

% adjust the page margins
\usepackage[scale=0.8]{geometry}
\usepackage{enumitem}
\usepackage[range-phrase=--, range-units=single, per-mode=symbol, range-phrase = ~--~]{siunitx}
\usepackage{mathtools}
\setlist[enumerate]{align=left}
%\setlength{\hintscolumnwidth}{3cm}                % if you want to change the width of the column with the dates
%\setlength{\makecvtitlenamewidth}{10cm}           % for the 'classic' style, if you want to force the width allocated to your name and avoid line breaks. be careful though, the length is normally calculated to avoid any overlap with your personal info; use this at your own typographical risks...

% todo change text below
% personal data
\name{}{Dr. Federico Tartarini}
% \title{Postdoctoral scholar}                               % optional, remove / comment the line if not wanted
\address{Senior Lecturer, The University of Sydney}{Wilkinson Building, 148 A36, Camperdown}{Sydney, NSW, Australia, 2050} % optional, remove / comment the line if not wanted; the "postcode city" and and "country" arguments can be omitted or provided empty
% \phone[mobile]{000-000-000-000}                   % optional, remove / comment the line if not wanted
% \phone[fixed]{+2~(345)~678~901}                    % optional, remove / comment the line if not wanted
% \phone[fax]{+3~(456)~789~012}                      % optional, remove / comment the line if not wanted
% todo change text below
\email{federico.tartarini@sydney.edu.au}                               % optional, remove / comment the line if not wanted
%\homepage{www.johndoe.com}                         % optional, remove / comment the line if not wanted
% \extrainfo{Berkeley Education Alliance for Research in Singapore}                 % optional, remove / comment the line if not wanted
%\photo[64pt][0.4pt]{picture}                       % optional, remove / comment the line if not wanted; '64pt' is the height the picture must be resized to, 0.4pt is the thickness of the frame around it (put it to 0pt for no frame) and 'picture' is the name of the picture file
%\quote{Some quote}                                 % optional, remove / comment the line if not wanted

% to show numerical labels in the bibliography (default is to show no labels); only useful if you make citations in your resume
%\makeatletter
%\renewcommand*{\bibliographyitemlabel}{\@biblabel{\arabic{enumiv}}}
%\makeatother
%\renewcommand*{\bibliographyitemlabel}{[\arabic{enumiv}]}% CONSIDER REPLACING THE ABOVE BY THIS

% bibliography with mutiple entries
%\usepackage{multibib}
%\newcites{book,misc}{{Books},{Others}}
%----------------------------------------------------------------------------------
%            content
%----------------------------------------------------------------------------------
\begin{document}
%-----       letter       ---------------------------------------------------------
% recipient data
% todo change text below
\recipient{Cover letter to the Editor}{of the journal of Building and Environment}
\date{\today}
% todo change text below
\opening{Dear Editors,}

\makelettertitle

% todo change text below
We submit an original research article titled ``Comparative analysis of PMV Models accuracy implemented in the ISO 7730:2005 and ASHRAE 55:2023'' for consideration for publication in Building and Environment.

The study compares the accuracy of the PMV models, as used in the ISO 7730:2005 standard, and its modified version, the PMV$_{CE}$ model, included in the ASHRAE 55:2023 standard.
The PMV$_{CE}$ was intentionally developed to estimate the effect of elevated air movement more accurately.
The two model formulations only differ when the air speed is greater than \qty{0.1}{\m\per\s}.
We determined the predictive accuracy of the PMV and PMV$_{CE}$ models by comparing their results to 49,282 thermal sensation votes available in the ASHRAE Global Thermal Comfort Database II v2.1.

We found that both models have low and similar prediction accuracy ($ \approx$\qty{34}{\percent}) and the PMV$_{CE}$ has higher bias than PMV in predicting thermal sensation in the subset of data with air speed greater than \qty{0.2}{\m\per\s}.
Both models have an accuracy lower or equal to \qty{7}{\percent} when predicting either `warm,' `hot,' or `cold.'
This value is lower than random guessing (\qty{14}{\percent}), meaning they are misleading the users.

Based on our results, we recommend restricting the use of the PMV only to values $\lvert \textrm{PMV} \lvert \leq 0.5$ since the PMV models cannot determine the degree to which people are dissatisfied.
Moreover, we recommend using the PMV model over the PMV$_{CE}$ model, as the latter is more complex and computationally intensive to run, while not being more accurate than the original PMV model.
Moreover, we challenge the assumption that individuals who feel `slightly warm' or `slightly cool' should be considered thermally comfortable.

We believe our findings will significantly contribute to the ongoing discourse on thermal comfort modeling and have practical applications in building design, engineering, and architecture.
Particularly because the PMV model is used by both the ISO 7730:2005 and ASHRAE 55:2023 standards.

We confirm that this manuscript has not been published elsewhere and is not under consideration by another journal.

All authors have approved the manuscript and agree with its submission to Building and Environment. 

We suggest the following reviewers:
\begin{itemize}
    \item Prof. Runming Yao, r.yao@reading.ac.uk
    \item Prof. Jørn Toftum, jt@byg.dtu.dk
    \item Prof. Shin-ici Tanabe, tanabe@tanabe.arch.waseda.ac.jp
    \item Assoc. Prof. Peixian Li, lipx@tongji.edu.cn
    \item Prof. Marcel Schweiker, mschweiker@ukaachen.de
\end{itemize}

We have no conflicts of interest to disclose.

Thank you for considering our manuscript for publication.
Please do not hesitate to contact me if I can be of any further assistance.
Please address all correspondence to Dr. Federico Tartarini at \href{mailto:federico.tartarini@sydney.edu.au}{federico.tartarini@sydney.edu.au}.
We look forward to your positive response.

\bigskip

% todo change text below
Dr. Federico Tartarini, PhD\\
Senior Lecturer\\
The University of Sydney\\

\end{document}