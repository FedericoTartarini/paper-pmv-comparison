%! Author = sbbfti
%! Date = 10/06/2020


\section{Methodology}\label{sec:methodology}

The \gls{db2} has \var{all_entries_db} independent recordings from various field investigations throughout the world and it is the largest of its kind.
\Ac{tsv} were collected using a \gls{rhrn} survey and environmental parameters were measured and logged while participants completed their survey.
The Comfort DB contains data collected by researchers in cross-sectional field studies that have been published in peer reviewed journals.
%We compared the results obtained by the \ac{pmv} and the \gls{pmv-ce} with the self-reported \ac{tsv}.
%    All the results were calculated using the function pmv_ppd [6] included in pythermalcomfort v1.10.0 a Python package for thermal comfort calculations.

\subsection{Data Processing and Cleaning}\label{subsec:data-processing-and-cleaning}
Not all researchers logged all the six parameters required to calculate the PMV.\@
We, therefore, decided to remove all entries that did not have any of the following variables: \ac{tdb}, \ac{rh}, \ac{v}, \ac{clo}, and \ac{met}.
These constraints filtered out approximately \qty{28}{\percent} of the data in the Comfort DB.\@  % todo check the above value
We decided to keep only entries that did not have \ac{tr} data but instead had the \ac{to} value.
Since \ac{tr} is a required input in the \ac{pmv} model, it was calculated using the following equation: 2\ac{to} – \ac{tdb}.
Alternatively, if not value of \ac{tr} was available we assumed \ac{tr} equal to \ac{tdb}.
We are aware that these assumptions may introduce an error, but removing samples without \ac{tr} would cause an additional \qty{44}{\percent} drop in the total entries in the database and there is evidence that in many conditions is possible to assume assumed \ac{tr} equal to tdb [7].  % todo check the above value and fix citation

Both the \gls{55} and \gls{7730} specify a set of applicability limits, we hence filtered out those data points which did not meet the inclusion criteria of both Standards.
The rationale is that the model accuracy should only be tested in within its applicability limits.
The limits we used to filter the data are as follows: \qty{10}{\celsius} $\leq$ \ac{tdb} $\leq$ \qty{30}{\celsius}, \qty{10}{\celsius} $\leq$ \ac{tr} $\leq$ \qty{40}{\celsius}, \qty{0}{\m\per\s} $\leq$ \ac{v} $\leq$ \qty{1}{\m\per\s}, \qty{0}{clo} $\leq$ \ac{clo} $\leq$ \qty{1.5}{clo}, 0 Pa $\leq$ water vapour partial pressure $\leq$ 2000 Pa, and \qty{1}{met} $\leq$ \ac{met} $\leq$ \qty{4}{met}.
We then calculated the adjusted total clothing insulation and relative airspeed as required by both Standards.
We used these inputs to calculate the \ac{pmv} values.  % todo check with marcel if this is required in his model too

The \ac{pmv} model is only applicable when its absolute value is lower than 2 \cite{Fanger1970, iso7730}.
However, since the thermal sensation was measured with at least a seven-point scale we decided also to show entries with |\ac{tsv}| or |\ac{pmv}| $\leq$ than \num{3.5}.
We used Python to analyse and plot the results.

We are committed to reproducible research hence we have shared the source code and the dataset we used publicly at this URL: \textbf{provide URL} so other users can test different assumptions.