%! Author = sbbfti
%! Date = 8/06/2022

% Preamble
\documentclass[11pt]{article}

\title{PMV model comparison}
\author{Federico Tartarini, Stefano Schiavon}

\usepackage{mypreamble}

\usepackage[printonlyused]{acronym}

\usepackage{glossaries}

\newglossaryentry{iso}{
      name = ISO 7730:2005 ,
      description = ISO Standard to calculate PMV and PPD,
}

\newglossaryentry{ashrae}{
      name = ASHRAE 55:2020 ,
      description = ASHRAE Standard to calculate PMV and PPD,
}

\begin{document}

    \maketitle

    \begin{abstract}
        The \ac{pmv} model developed by Fanger predicts thermal sensation.
        It is used in its original formulation in the \gls{iso}, while the \gls{ashrae} uses a modified version \acs{pmv-ce} to better account for air speeds.
        We compared their results against XXX thermal sensation votes collected in buildings.
    \end{abstract}

    \section*{Nomenclature}
    %! Author = federico
%! Date = 10/06/2020

\section*{Nomenclature}
\renewcommand{\baselinestretch}{0.75}\normalsize
\renewcommand{\aclabelfont}[1]{\textsc{\acsfont{#1}}}
\begin{acronym}[longest]

    \acro{t-db}[$t_{db}$]{dry-bulb air temperature\acroextra{, $^{\circ}$C}}
    \acro{t-wb}[$t_{wb}$]{wet-bulb air temperature\acroextra{, $^{\circ}$C}}
    \acro{ti}[$t_{i}$]{indoor air temperature\acroextra{, $^{\circ}$C}}
    \acro{tout}[$t_{out}$]{outdoor air temperature\acroextra{, $^{\circ}$C}}
    \acro{t-op}[$t_{o}$]{operative air temperature\acroextra{, $^{\circ}$C}}
    \acro{t-cl}[$t_{cl}$]{clothing temperature\acroextra{, $^{\circ}$C}}
    \acro{tg}[$t_{g}$]{globe temperature\acroextra{, $^{\circ}$C}}
    \acro{rh}[$RH$]{relative humidity\acroextra{, \%}}
    \acro{v}[$V$]{average air speed\acroextra{, m/s}}
    \acro{t-r}[$\overline{t_{r}}$]{mean radiant temperature\acroextra{, $^{\circ}$C}}
    \acro{clo}[$I_{cl}$]{total clothing insulation\acroextra{, clo}}
    \acro{i-cl}[$i_{cl}$]{permeation efficiency of water vapor through the clothing layer}
    \acro{met}[$M$]{rate of metabolic heat production\acroextra{, W/m\textsuperscript{2}}}

    \acro{pmv}[PMV]{Predicted Mean Vote}
    \acro{ppd}[PPD]{Predicted Percentage of Dissatisfied\acroextra{, \%}}
    \acro{set}[SET]{Standard Effective Temperature\acroextra{, $^{\circ}$C}}
    \acro{ce}[CE]{Cooling Effect\acroextra{, $^{\circ}$C}}
    \acro{phs}[PHS]{Predicted Heat Strain}

    \acro{BMS}[BMS]{Building Management System}
    \acro{HVAC}[HVAC]{Heating, Ventilation, and Air Conditioning}
    \acro{VAV}[VAV]{Variable Air Volume}
    \acro{AHU}[AHU]{Air Handling Unit}

    \acro{p-sk}[$p_{sk,s}$]{water vapor pressure at skin\acroextra{, kPa}}
    \acro{p-a}[$p_{a}$]{water vapor pressure in ambient air\acroextra{, kPa}}

    \acro{wmo}[WMO]{World Meteorological Organization}
    \acro{who}[WHO]{World Health Organization}
    \acro{cdc}[CDC]{Centers for Disease Control and Prevention}
    \acro{noaa}[NOAA]{National Oceanic and Atmospheric Administration}
    \acro{epa}[EPA]{United States Environmental Protection Agency}
    \acro{iea}[IEA]{International Energy Agency}
    \acro{un}[UN]{United Nations}

\end{acronym}
\renewcommand{\baselinestretch}{1}\normalsize


    \section{Introduction}\label{sec:introduction}
    In 1970, Fanger~\cite{Fanger1970} developed the \ac{pmv} model, which is now incorporated into the ISO 7730:2005 Standard~\cite{iso7730} in its original form.
    The \ac{pmv} is the most extensively used thermal comfort index for estimating the thermal sensation of a group of people sharing the same thermal environment.
    It is used both by researchers and practitioners.
    However, Cheung et al. (2019) demonstrated that the accuracy of the \ac{pmv} in predicting self-reported thermal sensation of people is only \qty{34}{\percent}~\cite{Cheung2019}.

    ASHRAE 55:2020 Standard utilizes a modified version of the PMV model [4] that calculates a Cooling Effect (CE) using the Standard Effective Temperature (SET) equation for air speeds higher than \qty{0.1}{\m\per\s}.
    The value of CE is then subtracted from both the measured dry-bulb air temperature (tdb) and \ac{tr} and these results become the input into the PMV model.
    Since the CE already accounts for convective heat losses from the person to the environment, the air speed used to calculate the PMV value is set to 0.1 m/s.

    \printbibliography

\end{document}