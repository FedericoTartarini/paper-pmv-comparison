\section{Introduction}\label{sec:introduction}
\mycite{Fanger1970} developed the \ac{pmv} model, which is now incorporated into the \gls{7730} standard~\cite{iso7730} in its original form.
The \ac{pmv} is an index that aims to predict the mean value of the thermal sensation votes (self-reported perceptions) of a large group of people on a sensation scale expressed from \num{-3} to \num{3} corresponding to the categories `cold,' `cool,' `slightly cool,' `neutral,' `slightly warm,' `warm,' and `hot.'~\cite{iso7730, ashrae552023}.
The \ac{pmv} model has been widely used worldwide by researchers and practitioners.
It is the most extensively used thermal comfort index.
A search within the article title, abstract, and keyword on Scopus with the search term `predicted mean vote' returns approximately \num{2050} documents.
Out of those \num{1200} are scientific peer-reviewed articles published in the research area of engineering, environmental science, social sciences, and energy.
This highlights this model's extensive adoption and use among the scientific community.
To this date, the \ac{pmv} remains to be the most utilized thermal comfort model even though several studies have highlighted that the \ac{pmv} has low accuracy in predicting thermal sensation votes~\cite{Cheung2019, Yao2022, Humphreys2002, doherty_evaluation_1988}.

\subsection{\ac{pmv-ce} model}\label{subsec:pmv-ce-limitations}
The \gls{55} standard uses a modified version of the original model, the \ac{pmv-ce}~\cite{ashrae552023}.
The best source to see how \ac{pmv-ce} works is the standard itself, but the standard does not explain why some changes to the \ac{pmv} inputs were implemented.
A partial justification of the model is mainly described in \mycite{arens_moving_2009} and secondarily in \mycite{yang_cooling_2015}.
The process used to calculate the \ac{pmv-ce} in the \gls{55} is summarized in the flowchart depicted in Figure~\ref{fig:flowchart_pmv_ce}.
\begin{figure}[!htb]
    \centering
    \begin{tikzpicture}[node distance=1cm and 3cm, transform shape]
        \node (start) [startstop] {\acs{pmv} calculation as per \gls{55}};
        \node (dec1) [decision, below of=start, yshift=-.25cm] {\acs{met} $> \qty{1}{met}$ };
        \node (pro1a) [process, right of=dec1, xshift=3.5cm] {\acs{v} = \acs{v}$+ 0.3 ($\acs{met}$-1)$};
        \node (dec2) [decision, below of=dec1, yshift=-0.75cm] {\acs{met} $> \qty{1.2}{met}$ };
        \node (pro2a) [process, right of=dec2, xshift=3.5cm] {\acs{clo} = \acs{clo}$(0.6 + 0.4/$\acs{met}$)$};
        \node (dec3) [decision, below of=dec2, yshift=-0.75cm] {\acs{v} $> \qty{0.1}{\m\per\s}$ };
        \node (end1) [startstop, below of=dec3, yshift=-.25cm, xshift=-4cm] {\acs{pmv}(\acs{tdb}, \acs{tr}, \acs{rh}, \acs{v}, \acs{met}, \acs{clo})};
        \node (pro3) [process, below of=dec3, yshift=-.25cm, xshift=3cm] {Calculate \acs{ce}};
        \node (end2) [startstop, below of=pro3, yshift=-.5cm] {\acs{pmv}(\acs{tdb} - \acs{ce}, \acs{tr} - \acs{ce}, \acs{rh}, \acs{v} = \qty{0.1}{\m\per\s}, \acs{met}, \acs{clo})};

        \draw [arrow] (start) -- (dec1);
        \draw [arrow] (dec1) -- node[above, pos=0.3] {Yes} (pro1a);
        \draw [arrow] (dec1) -- node[right, pos=0.3] {No} (dec2);
        \draw [arrow] (pro1a) -- ($(pro1a.south)+(0,-0.5)$) -| (dec2);
        \draw [arrow] (dec2) -- node[above, pos=0.3] {Yes} (pro2a);
        \draw [arrow] (dec2) -- node[right, pos=0.3] {No} (dec3);
        \draw [arrow] (pro2a) -- ($(pro2a.south)+(0,-0.5)$) -| (dec3);
        \draw [arrow] (dec3) -- ($(dec3.west)$) -| node[above, pos=0.3] {No} (end1);
        \draw [arrow] (dec3) -- ($(dec3.east)$) -| node[above, pos=0.3] {Yes} (pro3);
        \draw [arrow] (pro3) -- (end2);
    \end{tikzpicture}
    \caption{Flowchart depicting the steps for the calculation of the PMV following the \gls{55} standard.}
    \label{fig:flowchart_pmv_ce}
\end{figure}
In summary, when the \ac{v} exceeds \qty{0.1}{\m\per\s} the \gls{55} prescribes the use of the \ac{ce}.
\ac{v} is not the measured value by air speed sensors.
According to the standard, the measured value should be adjusted using this equation \ac{v} = \ac{v}$+ 0.3 ($\acs{met}$-1)$ if \ac{met}~$>$~\qty{1}{met}.
The value of \ac{ce} is calculated using the \ac{set} equation.
The \ac{ce} is then subtracted from both the \ac{tdb} and \ac{tr}.
The resulting values become the new inputs in the \ac{pmv} model.
Since the \ac{ce} accounts for convective and evaporative heat losses from the person to its environment, the value of \ac{v}, used to calculate the \ac{pmv}, is set to \qty{0.1}{\m\per\s}.
The other three input parameters (\ac{clo}, \ac{rh}, \ac{met}) remain unchanged.
Consequently, for a given thermal environment, results of the two \ac{pmv} formulations differ only when the value of \ac{v} is higher than \qty{0.1}{\m\per\s}.
To illustrate the extent of the differences between the outputs of the two models, we calculated the comfort regions ($\mid$\ac{pmv}$\mid \leq 0.5$) for different values of \ac{v} using the \ac{pmv} and \ac{pmv-ce} models assuming \ac{met}~=~\qty{1.2}{met} and \ac{clo}~=~\qty{0.5}{clo}.
We plotted the results in Figure~\ref{fig:comfort_regios_pmv_pmvce}.
\begin{figure}[!htb]
    \centering
    \includegraphics[width=1\textwidth]{figures/pmv_comfort_regions}
    \caption{Comfort regions ($|$\ac{pmv}$|$~$\leq$~\num{0.5}) calculated using \ac{pmv} and \ac{pmv-ce} models for two values of \ac{v}.
    We set \ac{met}~=~\qty{1.2}{met}, \ac{clo}~=~\qty{0.5}{clo}.
    \label{fig:comfort_regios_pmv_pmvce}}
\end{figure}
The results show that for \ac{v}~=~\qty{0.4}{\m\per\s} and \ac{rh}~=~\qty{50}{\percent} the comfort regions estimated using the \ac{pmv-ce} is \qty{0.7}{\celsius} wider than the one calculated using the \ac{pmv} and shifted towards warmer temperatures.
For \ac{v}~=~\qty{0.2}{\m\per\s}, the same difference is \qty{0.4}{\celsius}.

The rationale for the development of the \ac{pmv-ce} model is that the original \ac{pmv} formulation does accurately estimate the convective and evaporative heat losses from the skin to the environment~\cite{huang_applicability_2014}.
It is true that the \ac{pmv} does not accurately estimate the evaporative heat losses trough sweating.
The \ac{pmv} model assumes that sweating is minimal and constant, for a specific \ac{met}, hence if sweating would be required to maintain the body in thermal equilibrium the \ac{pmv} model return a positive \ac{pmv} value.
The human body, on the other hand, uses sweating to maintain a constant body temperature in `warm' environments.
However, despite this limitation of the \ac{pmv} model, we have identified several limitations of the \ac{pmv-ce} model, which we list below.
% \mycite{huang_applicability_2014}, using 19 averaged data points obtained from a set of experiments involving 30 subjects, showed the \ac{set} model could be used to estimate the thermal sensation with two different ways of estimating the airspeed. 
% They developed a model to relate the SET and the TSV. 
% They also plotted the \ac{pmv} versus the TSV and provided a visual analysis of its performance. They concluded that the SET is better than the \ac{pmv} model in predicting thermal sensation votes when subjects are exposed to air movements, but they did not report the performance of the \ac{pmv} model.
% Moreover, they did not compare the accuracy of the \ac{pmv-ce} with the \ac{pmv} model.

\begin{enumerate}
    \item We could not find a peer-reviewed scientific publication that quantified the accuracy improvements of the model as implemented in the ASHRAE standard over the original \ac{pmv} model.
    \item Although the \ac{ce} is used to better account for convective heat losses from the skin to the environment due to the air movement, the \ac{pmv-ce} model uses the same underlying equations as the \ac{pmv} model.
    For example, let's assume that a person is exposed to the following environmental conditions \ac{tdb}~=~\ac{tr}~=~\qty{29}{\celsius}, \ac{rh}~=~\qty{80}{\percent}, \ac{v}~=~\qty{0.15}{\m\per\s}, they have a \ac{met} of \qty{1}{met} and a \ac{clo} of \qty{0.5}{clo}.
    According to the Gagge's two-node model, which is used to calculate the \ac{set}, this hypothetical person would have \qty{30}{\percent} of their skin covered with sweat.
    Based on the results from \mycite{huang_applicability_2014}, the \ac{pmv} model, in this scenario, cannot accurately predict the \ac{tsv} since the person is sweating.
    However, the only difference between the \ac{pmv} and \ac{pmv-ce} in this scenario, is that the \ac{pmv-ce} accounts for the cooling effect due to the air movement (delta between \qty{0.15}{\m\per\s} and \qty{0.1}{\m\per\s}) which is equal to \qty{.17}{\celsius}.
    Even after reducing both \ac{tdb} and \ac{tr} by the \ac{ce}, the person is still estimated to be sweating.
    Therefore, in this scenario, the \ac{pmv-ce} model also cannot accurately predict the \ac{tsv}.
    \item The previous example, highlights another limitation of the \ac{pmv-ce}.
    The assumption of keeping \ac{rh} constant after adjusting \ac{tdb} and \ac{tr} is incorrect, since the value of \ac{rh} is dependent on \ac{tdb}.
    The \ac{rh} should be re-calculated, potentially assuming the humidity ratio to be constant, to reflect the new \ac{tdb} value.
    \item It should be noted that over the years, the critical value of \ac{v} has been changed from \qty{0.15}{\m\per\s} in the ASHRAE~55:2013~\cite{ASHRAE552013} to \qty{0.2}{\m\per\s} in the ASHRAE~55:2017~\cite{ASHRAE552017, arens_moving_2009} to \qty{0.1}{\m\per\s} in \gls{55}~\cite{ashrae552023}.
    \item The \ac{set} temperature which is used to calculate the \ac{ce}, is defined as the hypothetical isothermal environment at a \ac{rh} of \qty{50}{\percent}, \ac{v}~$\leq$~\qty{0.1}{\m\per\s}, and \ac{tr}~=~\ac{tdb} in which the total heat loss from the skin of an imaginary occupant wearing clothing, standardized for the activity concerned, is the same as that from a person in the actual environment with actual clothing and activity level~\cite{ashrae552023}.
    However, while \ac{v}, used to calculate the \ac{pmv-ce}, is set to \qty{0.1}{\m\per\s} the value of \ac{rh} and \ac{clo} are not adjusted to compensate for the fact that \ac{set} fixed the \ac{rh} to \qty{50}{\percent} and standardized the clothing to the activity.
    \item Calculating the \ac{pmv-ce} is more computationally intensive and complex since it requires the user to solve two heat balance equations, the \ac{pmv} and the \ac{set} model.
\end{enumerate}

\subsection{Other \ac{pmv} Formulations}\label{subsec:other-pmv-formulations}
Several other formulations of the \ac{pmv} model have been proposed.
Among the most notable formulations there are the \ac{pmvs}~\cite{GaggeSET}, \ac{pmvg}~\cite{GaggeSET}, \ac{epmv}~\cite{Toftum2002}, and \ac{athb}~\cite{Schweiker2022}.
\mycite{Yao2022} provides a comprehensive review of different thermal comfort models and compares and describes some of the above-mentioned models.
Here, we decided to focus on the \ac{pmv} and \ac{pmv-ce} models since they are incorporated in the most widely referenced thermal comfort standards worldwide (\gls{7730} and \gls{55}).

\subsection{Comparison of \ac{pmv} and \ac{pmv-ce}}\label{subsec:comparision-of-pmv-formulations}
To our knowledge, no previous study compared in detail the accuracy of the \ac{pmv} and \ac{pmv-ce} models.
Some notable works that tried to determine the accuracy of the \ac{pmv} model include the work from \mycite{doherty_evaluation_1988} who evaluated the ability of the \ac{pmv} and \ac{set} models to predict several physiological variables (i.e., skin temperature, core temperature, and skin wettedness) under a wide range of still air environments and metabolic rates.
They concluded that the \ac{pmv} model is accurate for simulations of resting subjects, but its accuracy decreases as a function of metabolic rate.
Humphreys and Nicol estimated the effects of measurement and formulation error on predicting thermal sensation using the \ac{pmv}~\cite{Humphreys2000}.
They used the ASHRAE Global Thermal Comfort database~I and determined that the measurement error and the error introduced by the \ac{pmv} model formulation had a similar and non-negligible contribution.
They also determined the validity of the \ac{pmv} for predicting comfort votes collected in field studies~\cite{Humphreys2002}.
They concluded that the \ac{pmv} range of applicability should be significantly reduced and it fails to predict the extent of thermal dissatisfaction of people in buildings.
The \ac{pmv} was free from bias only when it was used to predict thermal neutrality~\cite{Humphreys2002}.
\mycite{Cheung2019} determined the accuracy of the \ac{pmv} model, by comparing its results with the \ac{tsv} from the ASHRAE Global Thermal Comfort Database II.
They found that the thermal sensation predicted by the PMV model, on average, is one full thermal sensation scale unit away from the subject’s responses, confirming the results of~\mycite{Humphreys2002}.
\mycite{Cheung2019} also concluded that the accuracy of \ac{pmv} was only \qty{34}{\percent}, the model has a slightly higher prediction accuracy for sensation close to neutrality, but the accuracy declined towards either end of the thermal sensation scale, and it overestimated both hot and cold sensations.
They also found that the Predicted Percentage of Dissatisfied (PPD) failed to predict the percentage of unacceptable votes if the thermal sensation was predicted using the PMV model and suggested its removal from the thermal comfort standard.
These results were confirmed by the analysis of the Chinese Thermal Comfort database~\cite{du_evaluation_2022}.
\mycite{Yao2022} compared the \ac{pmv} and \ac{pmv-ce} models, however, their aim was primarily to compare these two formulations with other adaptive \ac{pmv} formulations, hence, they do not provide a detailed analysis on the prediction accuracy of the two models.
They focused significantly on how the models perform in different climates and when applied to people from different world regions, and their analysis only reports the mean bias of the different models.
This, as depicted by \mycite{Humphreys2000}, does not provide sufficient insights and information in determining which model is more accurate since it does not explain how the model performs over a wide range of environmental, personal conditions, and \ac{tsv}.
Reporting the classification accuracy of the \ac{pmv} formulations when people are grouped by their \ac{tsv} is particularly important for an unbalanced dataset like the \ac{db2} where most of the participants reported to be `neutral'.
Finally, \mycite{Yao2022} only reported the overall bias for the whole dataset, even though the two formulations only differ when the \ac{v} exceeds \qty{0.1}{\m\per\s}.

\subsection{Objectives}\label{subsec:aim-and-objectives}
Choosing between the \ac{pmv} and \ac{pmv-ce} is a source of confusion for researchers, educators and practitioners worldwide since both models are widely used in building codes, guidelines and certification programs.
For example, the WELL certification allows both compliance with \gls{7730} and \gls{55} standards.
In this paper, we compare the accuracy of the \ac{pmv} and \ac{pmv-ce} models used in the \gls{7730} and \gls{55} standards, respectively.
We used the \ac{tsv} recorded in the \acf{db2}.
We aim to determine which \ac{pmv} model is most accurate and the models' applicability limits.