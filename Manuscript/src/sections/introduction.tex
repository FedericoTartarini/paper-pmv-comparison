\section{Introduction}\label{sec:introduction}

In 1970, Fanger~\cite{Fanger1970} developed the \ac{pmv} model, which is now incorporated into the \gls{7730} Standard~\cite{iso7730} in its original form.
The \ac{pmv} model has been widely adopted worldwide both among researchers and practitioners.
It is the most extensively used thermal comfort index for estimating the thermal sensation of a group of people sharing the same thermal environment.
However, it has a relatively low accuracy in predicting self-reported thermal sensation of people~\cite{Cheung2019}.

To overcome this problem, the \gls{55} Standard uses a modified version of the original model developed by Fanger, the \gls{pmv-ce}~\cite{ashrae552020}.
For \ac{v} higher than \qty{0.1}{\m\per\s} the \gls{55} prescribes the use of the \ac{ce}.
The value of \ac{ce}, which is calculated using the \ac{set} equation, is subtracted from both the value of \ac{tdb} and \ac{tr}.
The resulting values become the new inputs into the \ac{pmv} model.
The \ac{ce} accounts for convective heat losses from the person to the environment, thus, the value of \ac{v}, used to calculate the \ac{pmv}, is set to \qty{0.1}{\m\per\s}.
While the other three input parameters (\ac{clo}, \ac{met}, and \ac{rh}) remain unchanged.
Consequently, for a given thermal environment, results from the two \ac{pmv} formulation differ when the value of \ac{v} is higher than \qty{0.1}{\m\per\s}.
This is a source of confusion for many users worldwide since both the \gls{7730} and the \gls{55} are the widely adopted and incorporated in many national and international building codes worldwide.
To further complicate this issue, over the course of the last \num{50} years, several other formulations of the \ac{pmv} model have been proposed.
Among the most notable formulations there are the:
\begin{enumerate}
    \item \gls{epmv}
    \item \gls{apmv}
    \item SET \ac{pmv}
    \item Gagge \ac{pmv}
    \item Marcel \ac{pmv}
\end{enumerate}

In 2002, Fanger, P. O. and Toftum, J. developed the \gls{epmv} an extension of the \ac{pmv}.
The \gls{epmv} includes an expectancy factor introduced for use in non-air-conditioned buildings in warm climates.
The authors argued that people adapt to the warm environment by decreasing their \ac{met} rate.
They, consequently, propose to reduce the metabolic rate used to calculate the \ac{pmv} by \qty{6.7}{\percent} for every scale unit of \ac{pmv} above neutral.
The resulting \ac{pmv} value is multiplied by the \ac{ef}.
The \ac{ef} for non-air-conditioned buildings can be determined by the length of warm weather throughout the year and whether such buildings may be compared to many others in the region that are air-conditioned.
If the temperature is hot all year or most of the year and there are no or few other air-conditioned buildings, \ac{ef} may be \num{0.5}, but if there are many other air-conditioned buildings, \ac{ef} may be \num{0.7}.
For non-air-conditioned buildings in locations where the temperature is warm solely during the summer and no or few buildings have air-conditioning \ac{ef} is \numrange{0.7}{0.8}, whereas it is \numrange{0.8}{.9} in areas where there are numerous air-conditioned buildings.
In regions with only brief periods of warm weather during the summer, the expectancy factor may be \numrange{0.9}{1}.
To better account for occupants adaptation Yao, R. et al. (2009) developed the Adaptive Predicted Mean
 Vote Model~\gls{apmv}.
They used a ``Black Box'' model to determine an adaptive coefficient which is in turn use to calculate the \gls{apmv} using the following equation: $\mathrm{aPMV} = \mathrm{PMV}/(1+\lambda \mathrm{PMV})$
The main issue with this approach is that the adaptive coefficient cannot be known a priori, but it is estimated based on the \ac{tsv} collected by a cohort of participants.

To our knowledge, no other study compared the accuracy all the aforementioned two aforementioned PMV models.
One study compared the accuracy of the \gls{pmv-ce} and the \ac{pmv} model.

This study aims to specifically fill this knowledge gap by comparing the results of the \ac{pmv} formulations developed over the course of the last years with self-reported \ac{tsv} contained in the ASHRAE Global Thermal Comfort Database II v2.1, referred to as the \gls{db2} thereafter.


