%! Author = sbbfti
%! Date = 10/06/2020

\section{Introduction}\label{sec:introduction}

    In 1970, Fanger~\cite{Fanger1970} developed the \ac{pmv} model, which is now incorporated into the \gls{7730} Standard~\cite{iso7730} in its original form.
    The \ac{pmv} model has been widely adopted worldwide both among researchers and practitioners.
    The \ac{pmv} is the most extensively used thermal comfort index for estimating the thermal sensation of a group of people sharing the same thermal environment.
    However, Cheung et al. (2019) demonstrated that the accuracy of the \ac{pmv} in predicting self-reported thermal sensation of people is only \qty{34}{\percent}~\cite{Cheung2019}.

    To overcome this problem, the \gls{55} Standard introduced a modified version of the \ac{pmv}, the \gls{pmv-ce}~\cite{ashrae552020}.
    For \ac{v} higher than \qty{0.1}{\m\per\s} the \gls{55} prescribes the use of the \ac{ce}.
    The value of \ac{ce}, calculated using the \ac{set} equation, is subtracted from both the value of \ac{tdb} and \ac{tr} and these results become the input into the \ac{pmv} model.
    The \ac{ce} already accounts for convective heat losses from the person to the environment, thus, the value of \ac{v}, used to calculate the \ac{pmv}, is set to \qty{0.1}{\m\per\s}.
    The other three input parameters (\ac{clo}, \ac{met}, and \ac{rh}) are not modified.
    Consequently, given the same thermal environment and same occupants, results from the two \ac{pmv} formulation differ when the value of \ac{v} is higher than \qty{0.1}{\m\per\s}.
    This may be a source of confusion for many users worldwide since both the \gls{7730} and the \gls{55} are the widely adopted and incorporated in many national building codes worldwide.
    In addition, over the course of the last \num{50} years, other formulations of the \ac{pmv} model have been formulated by other researchers.
    To our knowledge, no other study compared the accuracy all the aforementioned  two aforementioned PMV models.
    One study compared the accuracy of the \ac{pmv-ce} and the \ac{pmv} model.

    This study aims to specifically fill this knowledge gap by comparing the results of the \ac{pmv} formulations developed over the course of the last years with self-reported \ac{tsv} contained in the ASHRAE Global Thermal Comfort Database II v2.1, referred to as the \gls{db2} thereafter.


