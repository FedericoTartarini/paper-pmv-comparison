\documentclass[a4paper, 10pt]{letter}

% package to highlight text
\usepackage{color,soul}

\usepackage{hyperref}
\hypersetup{
    colorlinks=false,
}

\usepackage{graphicx}

\usepackage[textsize=scriptsize]{todonotes}

% Name of sender
\name{Federico Tartarini}

% Signature of the sender
\signature{Federico Tartarini}

\newcommand{\response}[1]{\textcolor{blue}{\textbf{Author response:} #1}}

%--------------------------------------------------------------------------%

\begin{document}

% Name and address of the receiver
    \begin{letter}
    {
        Response to reviewers' comments.
    }

% Opening statement
        \opening{Dear Athanasios Tzempelikos and Reviewers,}

% Letter body

        We would like to thank you for your time and your valuable feedback.
        Please find below our answers to all your comments.

        \textbf{Reviewer 1}
        Dear Authors,
        This manuscript is well-written and showcases your strong capabilities in addressing thermal comfort issues.
        I have just a few comments to share with you Wishing you a Merry Christmas and a Happy New Year.

        \begin{enumerate}

            \item P4 L17.
            Icl is the basic clothing insulation.
            Total clothing insulation accounts for the air insulation and the clothing area factor (see ISO 9920).

            \todo[color=yellow]{Federico - check what the PMV model uses for clothing insulation as input and change the text accordingly}
            \response{
                Here our response
            }

            \item P4 L34-38.
            ``The rationale for the development of the PMVCE model is that the original PMV
            formulation does not accurately estimate convective and evaporative heat losses from
            the skin to the environment [10].`` If this is the rationale, why CE is subtracted from both
            the dry-bulb air temperature (tdb) and mean radiant temperature (tr)?
            Please provide evidence.

            \todo{Stefano - is the following answer true? Did not you pulish the paper which explains why the CE is subtracted from both the tdb and tr?}
            \response{
                Thank you for pointing this issue out, this is clearly a important issue about the PMVCE model.
                Unfortunately we are not certain why the PMVCE subtracts the cooling effect from both the air and mean radiant temperature, as we were not involved in the model's development.
                In Section 1.1 we have tried to summarise all the knowledge we have on the PMVCE model, in particular we stated explained that while the ASHRAE 55:2023 standard explains how the PMVCE works.
                There is no peer-reviewed publication that quantifies the accuracy improvements of the model as implemented in the ASHRAE standard over the original PMV model.
                Arens et al. (2009) and Yang et al. (2015) provide a partial justification for the PMVCE model, they do not fully explain why these specific changes to the PMV inputs were made.
            }

            \item P7 L28-32.
            The authors address (also in methods and in the section 3.3) a highly relevant issue: the impact of microclimatic measurements on the accurate evaluation of thermal comfort.
            Unfortunately, thermal comfort databases suffer from significant biases due to several factors, including the absence of standardized measurement protocols (such as representative positions and height above the ground), incorrect or arbitrary assumptions (for instance, the belief in uniform and stable conditions), the use of inaccurate measurement devices, deficiencies in software, and the failure to assess clothing insulation.
            As a result, the PMV (Predicted Mean Vote) value can appear random, making it difficult to evaluate the category of indoor environmental quality.
            These issues have been previously discussed (e.g., in 10.1016/j.buildenv.2011.01.001) and should be emphasized more prominently in the current investigation.
            Such problems impact all PMV-related models, including adaptive models, where the operative temperature is often assumed to be the same as the air temperature.

            \todo[color=yellow]{Federico - add the following sentences and reference them in the comment below}
            \response{
                We agree with the reviewer that the accuracy of the PMV model is influenced by the quality of the input data.
                We have added a sentence in the introduction to highlight the importance of the quality of the input data.
                We have also added a sentence in the limitations section to acknowledge the limitations of the PMV model.
            }

            \item The database used for this investigation also includes high air velocity measurements (v>0.2 m/s).
            However, this does not rule out the possibility of local discomfort caused by draughts.
            What about the impact of these draughts on the overall thermal comfort as quantified by the Thermal Sensation Vote (TSV)?
            Are the authors confident that the TSV values collected at high air velocity are not influenced by local discomfort?
            Please provide a brief explanation.

            \todo{Stefano shall we consider adding a sentence about the local discomfort in the paper?
            Can we say that this is potentially a limitation of the PMV model?}
            \response{
                We agree with the reviewer that draughts can be the cause local discomfort and consequently influence the overall thermal comfort.
                However, the PMV model does not estimate local discomfort and can be used to predict thermal sensation in a space with air speeds up to 0.2 m/s.
                This, with other factors, can lead to a discrepancy between the PMV and the TSV.
            }

        \end{enumerate}

        \clearpage

        \textbf{Reviewer 2}
        This article conducts a comparative analysis of the accuracy of the PMV model implemented in ISO 7730:2005 and ASHRAE 55:2023 using data from the ASHRAE 2 database.
        The research has good practical significance, but there are some issues that need further clarification:

        \begin{enumerate}
            \item The full text discusses the PMV model in ISO 7730:2005 and the PMVCE model in ASHRAE 55:2023, that is, the PMV models in the standards.
            When Professor Fanger initially proposed the PMV model, there were certain applicable conditions.
            Does the article aim to explore the accuracy of the PMV model, or the inappropriate application of this model in the two standards?

            \todo{Stefano - I cannot fully understand the question, can you help me?}
            \response{This manuscript aims to explore the accuracy of the PMV model implemented in the ISO 7730:2005 and ASHRAE 55:2023 standards.
            We did this by comparing the results of the PMV model with the thermal sensation votes collected in the ASHRAE Global Thermal Comfort Database II v2.1.
            In addition to comparing the accuracy of the two PMV models, we also discussed in the paper that the PMV model is inappropriately used by some authors to predict thermal sensation of individuals outside its range of applicability.
            As a consequence, we recommend limiting the applicability of the PMV model to the range specified in the Table 2 of the manuscript.
            }

            \item It is stated in the abstract that `` The ISO7730:2005 and ASHRAE55:2023 are the most widely referenced thermal comfort standards worldwide, and their different PMV formulations are a source of confusion.``
            However, it does not actually clarify what kind of confusion has been caused.

            \todo[color=yellow]{Federico - add the following sentences and reference it in the comment below}
            \response{Thank you for pointing this out.
            We should have explained this better in the manuscript.
            The confusion arises from the fact that the two standards use different PMV formulations and the results of the two models can differ.
            Consequently, for a given thermal environment, a user may obtain different results depending on the standard used.
            This can lead to confusion when comparing results from different studies or when trying to apply the results in practice.
            We have added a sentence in the introduction to clarify this point.
            }

            \item In lines 32--34, ``Consequently, for a given thermal environment, results of the two PMV formulations differ only when the value of V is higher than 0.1m/s.``
            In the calculation of PMVCE, the metabolic rate and clothing thermal resistance will also be corrected.
            So why does it claim that there are differences in the calculation results of the two models only when the ``V`` is greater than 0.1?
            Besides, does the ``V`` that the author wants to express refer to the original measured value or the adjusted V value?

            \todo[color=yellow]{Federico - add the following sentences and reference it in the comment below}
            \response{
            Please let us clarify this point.
            1) Neither the PMV nor the PMVCE model correct the metabolic rate.
            2) Both the PMV and PMVCE models correct the clothing insulation based on the metabolic rate.
            3) Both the PMV and PMVCE models correct the air speed based on the metabolic rate using the same equation.
                The value of V refers to the adjusted air speed.
                We have spedified this in the nomeclature section of the manuscript.
                However, we strongly agree with the reviewer that this point is not clear and this issue arises from how the ASHRAE 55:2023 standard is written.
                The authors of the ASHRAE 55:2023 standard voted not to have two variables one for the measured air speed and one for the adjusted air speed.
                We have added a sentence in the manuscript to raise this issue about the wording of the ASHRAE 55:2023 standard.
            }

            \item In the description of limitations on line 50, the first item does not seem to be a limitation of the model itself.
            It is recommended to include it in the statement of the research purpose and significance.

            \response{Here our reply}

            \item In the data selection, there is no selection of environmental types.
            If the aim is to test the accuracy of the PMV model, were data from naturally ventilated environments, outdoor environments, etc.\ also selected?

            \response{Here our reply}

            \item In the data selection in lines 168--171, it has been stated that ``the PMV should only be used when its absolute value is lower than 2``.
            So why were data with |PMV| $\leq$ 3.5 retained?
            Regarding the retention of TSV data, it is said that ``thermal sensation was measured with at least a seven-point scale``, which indicates that there are different scales.
            When the scale sizes are different, the same number represents different meanings.
            Is it reasonable to retain data in the form of |TSV| $\leq$ 3?

            \response{Here our reply}

            \item There are problems in the analysis of Figure 5i.
            The figure shows that subjects with slightly warm and slightly cold thermal sensations have the intention to further change the environment, but this does not mean that they are in an uncomfortable state.
            Even when the thermal sensation is neutral, people with different thermal preferences may still have expectations for environmental changes.
            The expectation of changing the environment and the evaluation of the current environment should not be confused.
            It is recommended to re-analyze after clarifying the meanings of thermal comfort, thermal sensation, and thermal preference.

            \response{Here our reply}

            \item It should be noted that the PMV model proposed by Fanger is applicable to steady-state indoor thermal environments that do not deviate too much from neutrality.
            Therefore, it is unreasonable to use overheated or overcooled data to verify the accuracy of the PMV model, and the demonstrated accuracy will also be relatively low.

            \response{Here our reply}

            \item For highlights 4--5, whether in field studies or laboratory studies, PMV values generally ranging from -1 to 1 are defined as thermal comfort, which is appropriate for steady-state indoor environments close to neutrality.
            However, if it is narrowed down to -0.5 to 0.5, the thermal comfort range is greatly reduced, which will increase the building operation energy consumption and is not recommended.

            \response{Here our reply}

            \item Some conclusion contents lack data and theoretical support.

            \response{Here our reply}

        \end{enumerate}

        \clearpage

        \textbf{Reviewer 3}
        The manuscript presents an excellent analysis of the discrepancies between two versions of the PMV model and evaluates their prediction accuracy.
        The analysis is informative and definitely relevant for researchers and professionals within thermal comfort.
        The structure of the manuscript and the presentation of the topic are well laid out.

        \begin{enumerate}
            \item I have a only a specific few comments, which are listed below.
            However, my main concern deals with the way the records representing individuals in the comfort DB are used to first calculate PMV for individuals and then compare with individual TSVs.
            As the authors write ``The intended aim of the PMV model is not to accurately predict each individual thermal response from participants.
            The PMV model was developed to predict the average thermal sensation of an undefined large group of occupants sharing the same environment.`` But isn't this analysis relying on calculation of each individual TSV without considering that this is not what the model should be used for?
            The approach is the same across a wealth of studies that examine the accuracy of the model, but that does not make it more correct.
            I actually doubt that the DB can be reliably used to estimate the prediction accuracy, as there is no information of the mean TSV of groups of people exposed to the same thermal environment.

            \response{Here our response}

            \item Although less critical, another general comment is that it is not clear what is meant by low predicition accuracy, as is written already in the abstract - when is the prediction accurqacy low?
            We aim to estimate what people perceive, which seriously suffers from individual differences and many others challenges.
            What is the prediction accuracy of related models?
            Or is the low prediction accuracy an outcome of using the prediction model wrongly (see first general comment).

            \response{Here our response}

            \item Ln. 103 - what is meant by formulation error?
            Unsuitable equations or constants that are wrong?

            \response{Here our response}

            \item Sources of error.
            Order of importance - is it really the heat transfer calculation that is the most important - not the attempt to calculate thermal sensation based on physical variables

            \response{Here our response}

            \item ``Additionally, the PMV model erroneously assumes that the human body is always losing or gaining heat from its surrounding environment.
            In reality, under most indoor conditions, the body activates control mechanisms to maintain a stable core temperature [24].`` Indeeed, the thermoregulatory system maintains a stable body temperature, but the body still exchanges heat with the environment, so suggest to revise this sentence.

            \response{Here our response}

            \item P 23 ln 416.
            Calculation of heat loss by sweating.
            Actually, this equation applies to people in thermal comfort, which should probably be added to the statement, so heat lost due to sweating does not appear to be related only with met rate.

            \response{Here our response}

            \item ``The PMV model uses the overall heat losses or gains to calculate a PMV value which should represent a TSV vote.`` This is wrong - the PMV was never meant to represent a TSV vote, but the average of TSV votes of a large group of people.

            \response{Here our response}

            \item Comment on measurement of air speed.
            ISO7726, which is used as a guide for measurements in ISO7730, recommends measurement of air speed only at abdomen level in homogeneous environments in both class C (comfort) and S (stress).
            The air speed measured in this height should be used to calculate the PMV\@.
            When comparing 7730 and 55, especially with emphasis on air speed, this should probably be taken into account.

            \response{Here our response}

        \end{enumerate}

        Kind regards,

        \vspace*{5px}

        Federico Tartarini

    \end{letter}
\end{document}
%-----------------------------------------------------------------------------%