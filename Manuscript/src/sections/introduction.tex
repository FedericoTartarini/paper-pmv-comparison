\section{Introduction}\label{sec:introduction}

In 1970, Fanger~\cite{Fanger1970} developed the \ac{pmv} model, which is now incorporated into the \gls{7730} Standard~\cite{iso7730} in its original form.
% todo also mention the other standards that use the PMV The PMV index has been widely used in national and international standards and guidance ASHRAE 55 [4], EN 15251 (new version EN 16798 [5]) [6], ISO 7730 [1], CIBSE Guide A [7] and GB/T 50785 [110] as  shown in  Table 5. This sentence is from Yao 2022.
The \ac{pmv} model has been widely adopted worldwide both among researchers and practitioners.
It is the most extensively used thermal comfort index for estimating the thermal sensation of a group of people sharing the same thermal environment.
A search within article title, abstract, and keyword on Scopus with the search term ``predicted mean vote'' returns approximately 1600 documents.
Out of those 1082 are scientific peer-review articles published in the research area of engineering, environmental science, social sciences, and energy.
This highlights the extensive adaption and use of this model among the scientific community.

To this date, the \ac{pmv} still remains to be the most utilized thermal comfort model despite the fact that several research papers have highlighted that the \ac{pmv} has a relatively low accuracy in predicting self-reported thermal sensation of people~\cite{Cheung2019}.  % todo cite more papers

To overcome this problem, the \gls{55} Standard uses a modified version of the original model developed by Fanger, the \gls{pmv-ce}~\cite{ashrae552020}.
The process used to calculate the \gls{pmv-ce} in the \gls{55} is summarized in the flowchart depicted in Figure~\ref{fig:flowchart_pmv_ce}.
For \ac{v} higher than \qty{0.1}{\m\per\s} the \gls{55} prescribes the use of the \ac{ce}.
The value of \ac{ce}, which is calculated using the \ac{set} equation, is then subtracted from both the value of \ac{tdb} and \ac{tr}.
The resulting values become the new inputs into the \ac{pmv} model.
The \ac{ce} accounts for convective heat losses from the person to the environment, thus, the value of \ac{v}, used to calculate the \ac{pmv}, is set to \qty{0.1}{\m\per\s}.
While the other three input parameters (\ac{clo}, \ac{met}, and \ac{rh}) remain unchanged.
Consequently, for a given thermal environment, results from the two \ac{pmv} formulation differ when the value of \ac{v} is higher than \qty{0.1}{\m\per\s}.
The rationale for this approach is that the original formulation of the \ac{pmv} model cannot accurately estimate latent and sensible heat losses from the skin to the environment when \ac{v} is higher than \qty{0.1}{\m\per\s}.
This method, however, has its own limitation and throughout the years the critical value of \ac{v} has been changed, from \qty{0.15}{\m\per\s} in \mycite{Schiavon2014} to \qty{0.2}{\m\per\s} in the ASHRAE 55--2017~\cite{ASHRAE552017} to \qty{0.1}{\m\per\s} in \gls{55}~\cite{ashrae552020}.

\begin{figure}[!htb]
    \centering
    \label{fig:flowchart_pmv_ce}

    \begin{tikzpicture}[node distance=1cm]
        \node (start) [startstop] {\ac{pmv} calculation as per \gls{55}};
        \node (dec1) [decision, below of=start, yshift=-.25cm] {\ac{met} $> \qty{1}{met}$ };
        \node (pro1a) [process, right of=dec1, xshift=4cm] {\ac{v} = \ac{v}$+ 0.3 ($\ac{met}$-1)$};
        \node (dec2) [decision, below of=dec1, yshift=-0.75cm] {\ac{met} $> \qty{1.2}{met}$ };
        \node (pro2a) [process, right of=dec2, xshift=4cm] {\ac{clo} = \ac{clo}$(0.6 + 0.4/$\ac{met}$)$};
        \node (dec3) [decision, below of=dec2, yshift=-0.75cm] {\ac{v} $> \qty{0.1}{\m\per\s}$ };
        \node (end1) [startstop, below of=dec3, yshift=-.25cm, xshift=-4cm] {\ac{pmv}(\ac{tdb}, \ac{tr}, \ac{rh}, \ac{v}, \ac{met}, \ac{clo})};
        \node (pro3) [process_large, below of=dec3, yshift=-.25cm, xshift=4cm] {Calculate \ac{ce}};
        \node (end2) [startstop, below of=pro3, yshift=-.5cm] {\ac{pmv}(\ac{tdb} - \ac{ce}, \ac{tr} - \ac{ce}, \ac{rh}, \ac{v} = \qty{0.1}{\m\per\s}, \ac{met}, \ac{clo})};

        \draw [arrow] (start) -- (dec1);
        \draw [arrow] (dec1) -- node[above,pos=0.3] {Yes} (pro1a);
        \draw [arrow] (dec1) -- node[right,pos=0.3] {No} (dec2);
        \draw [arrow] (pro1a) -- ($(pro1a.south)+(0,-0.5)$) -| (dec2);
        \draw [arrow] (dec2) -- node[above,pos=0.3] {Yes} (pro2a);
        \draw [arrow] (dec2) -- node[right,pos=0.3] {No} (dec3);
        \draw [arrow] (pro2a) -- ($(pro2a.south)+(0,-0.5)$) -| (dec3);
        \draw [arrow] (dec3) -- ($(dec3.west)$) -| node[above,pos=0.3] {No} (end1);
        \draw [arrow] (dec3) -- ($(dec3.east)$) -| node[above,pos=0.3] {Yes} (pro3);
        \draw [arrow] (pro3) -- (end2);
    \end{tikzpicture}
\end{figure}

This is a source of confusion for many users worldwide since both the \gls{7730} and the \gls{55} are the widely adopted and incorporated in many national and international building codes worldwide.

\subsection{Other PMV formulations}\label{subsec:other-pmv-formulations}
To further complicate this issue, several other formulations of the \ac{pmv} model have been proposed worldwide.
Among the most notable formulations there are the: \gls{pmvs}, \gls{pmvg}, \gls{epmv}, \gls{apmv}, and \ac{athb}.
\mycite{Yao2022} provides a comprehensive review of the difference \ac{pmv} formulations and the adaptive heat balance models based on \ac{pmv}.
Hence, we invite our readers to read \mycite{Yao2022} paper to learn more about all the different variations that have been proposed.

%% todo describe pmv SET
%% todo describe pmv Gagge
%
%In 2002, Fanger, P. O. and Toftum, J. developed the \gls{epmv} an extension of the \ac{pmv}.
%The \gls{epmv} includes an expectancy factor introduced for use in non-air-conditioned buildings in warm climates.
%The authors argued that people adapt to the warm environment by decreasing their \ac{met} rate.
%They, consequently, propose to reduce the metabolic rate used to calculate the \ac{pmv} by \qty{6.7}{\percent} for every scale unit of \ac{pmv} above neutral.
%The resulting \ac{pmv} value is multiplied by the \ac{ef}.
%The \ac{ef} for non-air-conditioned buildings can be determined by the length of warm weather throughout the year and whether such buildings may be compared to many others in the region that are air-conditioned.
%If the temperature is hot all year or most of the year and there are no or few other air-conditioned buildings, \ac{ef} may be \num{0.5}, but if there are many other air-conditioned buildings, \ac{ef} may be \num{0.7}.
%For non-air-conditioned buildings in locations where the temperature is warm solely during the summer and no or few buildings have air-conditioning \ac{ef} is \numrange{0.7}{0.8}, whereas it is \numrange{0.8}{.9} in areas where there are numerous air-conditioned buildings.
%In regions with only brief periods of warm weather during the summer, the expectancy factor may be \numrange{0.9}{1}.
%This arbitrary and non-scientific definition of the expectancy factor make the \gls{epmv} difficult to calculate and non-generalizable.
%Consequently, the \gls{epmv} has not gained any widespread adoption.
%
%Yao, R. et al. (2009) developed the Adaptive Predicted Mean Vote Model~\gls{apmv} to account for occupants adaptation.
%They used a ``Black Box'' model to estimate an adaptive coefficient which is in turn use to calculate the \gls{apmv} using the following equation: $\mathrm{aPMV} = \mathrm{PMV}/(1+\lambda \mathrm{PMV})$
%The main issue with this approach is that the adaptive coefficient cannot be known a priori, but it is estimated based on the \ac{tsv} collected by a cohort of participants.
%This approach cannot be, therefore, generalizable and similarly to the \gls{epmv} model the \gls{apmv} is rarely used in thermal comfort research.
%
%% todo describe marcel's work

\subsection{Comparision of PMV formulations}\label{subsec:comparision-of-pmv-formulations}
To our knowledge, no previous study compared in detail the accuracy of the \ac{pmv} and \ac{pmv-ce} models which are currently used in the \gls{7730} and \gls{55} Standards.

Some notable works include, the work from \mycite{Doherty1988} who evaluated the ability of the \ac{pmv} and \ac{set} models to predict several physiological variables (i.e., skin temperature, core temperature, and skin wettedness) under a wide range of still air environments and metabolic rates.
They concluded that the \ac{pmv} model is accurate for simulations of resting subjects but its accuracy decreases as a function of metabolic rate.
Humphreys and Nicol estimated the effects of measurement and formulation error on predicting thermal sensation using the \ac{pmv}~\cite{Humphreys2000}.
They used the ASHRAE Thermal Comfort database v1.0 and determined that the measurement error and the error introduced by the \ac{pmv} model formulation had a similar and non-negligible contribution.
Humphreys and Nicol also determined the validity of the \ac{pmv} for predicting comfort votes collected in field studies~\cite{Humphreys2002}.
They concluded that the \ac{pmv} range of applicability should be significantly reduced and it fails to predict the extent of thermal dissatisfaction of people in buildings.
The \ac{pmv} was free from bias only when it was used to predict thermal neutrality.
While, \mycite{Cheung2019} determined the accuracy of the \ac{pmv} model, by comparing its results with the \ac{tsv} from the ASHRAE Global Thermal Comfort Database II.
Finally, while \mycite{Yao2022} compare the \ac{pmv} and \ac{pmv-ce} models, their aim was primarily to compare these two formulations with other adaptive \ac{pmv} formulations, hence, they do not provide a detailed analysis on the prediction accuracy of the two models.
In particular, they focus significantly how the models perform in different climates and when applied to people from different world regions, but for example they fail to report how the \ac{pmv} accuracy varies as a function of \ac{tsv} and how the bias of the model varies as a function of different environmental parameters.

\subsection{Aim and Objectives}\label{subsec:aim-and-objectives}
This study aims to specifically fill this knowledge gap by comparing the results of the \ac{pmv} and \ac{pmv-ce} models which are currently used in the \gls{7730} and \gls{55} Standards.
We compared the results against the self-reported \ac{tsv} contained in the ASHRAE Global Thermal Comfort Database II v2.1, referred to as the \gls{db2} thereafter.
Our results have significant and profound implications for both researchers, practitioners, and policymakers since it provides quantitative evidence on which formulation of the \ac{pmv} model is more accurate.
This in turn will allow them to take an informed decision on which model it should be used in National and International Standards to predict thermal sensation.
Policymakers would also benefit from our results since they can use them to review the current versions of the \gls{7730} and \gls{55} Standards.