%! Author = sbbfti
%! Date = 10/06/2020

\begin{frontmatter}

    \title{PMV formulations: which one is the most accurate?}

    \author[label1,label2]{Federico Tartarini\corref{mycorrespondingauthor}}
    \ead{federicotartarini@berkeley.edu}
    \author[label3]{Stefano Schiavon}

    \address[label1]{Berkeley Education Alliance for Research in Singapore, Singapore}
    \address[label2]{Heat and Health Research Incubator, Faculty of Health and Medicine, University of Sydney, Sydney, AU}
    \address[label3]{Center for the Built Environment, University of California, Berkeley, CA, USA}

    \cortext[mycorrespondingauthor]{Corresponding author}

    \begin{abstract}
        The primary objectives of buildings is to shelter pople from the elements and provide a comfortable thermal environment.
        The \ac{pmv} model was therefore developed by Fanger more than 50 years ago to predict the thermal sensation of a large group of occupants sharing the same thermal environment.
        This should enable to determine the optimal temperature range to be maintained indoors to satisfy the great majority of the occupants.
        The \ac{pmv} has seen widespread adoption and is and has been used by practitioners to design and operate buildings and by scientists to predict thermal sensation.
        It is still currently used in its original form in the \gls{7730} despite the fact that several research studies have highlighted that the \ac{pmv} model has a low prediction accuracy, especially when it tries to predict the thermal sensation of people who are not thermally neutral.
        To compensate for that, several other formulations have been proposed, one of which one is currently used in the \gls{55}.
        Being the \gls{7730} and \gls{55} the most widely used Standards worldwide, this is a source of confusion for practitioners and researchers who find it difficult to decide which formulation to use.
        We aim to solve this problem by comparing the accuracy of the two main \ac{pmv} formulations, currently being used a reference in the above-mentioned standard, against \var{entries_db_used} thermal sensation votes collected in buildings.
        % We found that for V>0.1m/s the \gls{pmv-ce} is significantly more accurate (48%) than the PMV (42%) in predicting thermal neutrality.
        % In 95% of the cases, the error was less than one point.
        % The PMV and PMVCE accuracy were 35% and 36% for |PMV|≤2 and 43% and 44% for |PMV|≤1, respectively.
        % Neither PMV model could accurately predict ‘cold’, ‘cool’, ‘warm’, or ‘hot’ thermal sensations.
        % Consequently, the PMV should only be used in conditions close to neutrality within the range |PMV|≤1.
         We open-sourced the source code we used to calculate the \ac{pmv} and we used to analyse the data.
    \end{abstract}

    \begin{keyword}
        Personal thermal comfort model \sep Ecological momentary assessment \sep Skin temperature \sep Machine learning \sep Internet of Things (IoT)
    \end{keyword}

\end{frontmatter}
