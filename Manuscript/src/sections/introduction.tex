\section{Introduction}\label{sec:introduction}

In 1970, Fanger~\cite{Fanger1970} developed the \ac{pmv} model, which is now incorporated into the \gls{7730} Standard~\cite{iso7730} in its original form.
The \ac{pmv} model has been widely adopted worldwide both among researchers and practitioners.
It is the most extensively used thermal comfort index for estimating the thermal sensation of a group of people sharing the same thermal environment.
However, it has a relatively low accuracy in predicting self-reported thermal sensation of people~\cite{Cheung2019}.

To overcome this problem, the \gls{55} Standard uses a modified version of the original model developed by Fanger, the \gls{pmv-ce}~\cite{ashrae552020}.
The process used by the Standard is summarized in the flowchart depicted in Figure~\ref{fig:flowchart_pmv_ce}.
For \ac{v} higher than \qty{0.1}{\m\per\s} the \gls{55} prescribes the use of the \ac{ce}.
The value of \ac{ce}, which is calculated using the \ac{set} equation, is subtracted from both the value of \ac{tdb} and \ac{tr}.
The resulting values become the new inputs into the \ac{pmv} model.
The \ac{ce} accounts for convective heat losses from the person to the environment, thus, the value of \ac{v}, used to calculate the \ac{pmv}, is set to \qty{0.1}{\m\per\s}.
While the other three input parameters (\ac{clo}, \ac{met}, and \ac{rh}) remain unchanged.
Consequently, for a given thermal environment, results from the two \ac{pmv} formulation differ when the value of \ac{v} is higher than \qty{0.1}{\m\per\s}.

\begin{figure}[!htb]
    \centering

    \begin{tikzpicture}[node distance=1cm]
        \node (start) [startstop] {\ac{pmv} calculation as per \gls{55}};
        \node (dec1) [decision, below of=start, yshift=-.25cm] {\ac{met} $> \qty{1}{met}$ };
        \node (pro1a) [process, right of=dec1, xshift=4cm] {\ac{v} = \ac{v}$+ 0.3 ($\ac{met}$-1)$};
        \node (dec2) [decision, below of=dec1, yshift=-0.75cm] {\ac{met} $> \qty{1.2}{met}$ };
        \node (pro2a) [process, right of=dec2, xshift=4cm] {\ac{clo} = \ac{clo}$(0.6 + 0.4/$\ac{met}$)$};
        \node (dec3) [decision, below of=dec2, yshift=-0.75cm] {\ac{v} $> \qty{0.1}{\m\per\s}$ };
        \node (end1) [startstop, below of=dec3, yshift=-.25cm, xshift=-4cm] {\ac{pmv}(\ac{tdb}, \ac{tr}, \ac{rh}, \ac{v}, \ac{met}, \ac{clo})};
        \node (pro3) [process_large, below of=dec3, yshift=-.25cm, xshift=4cm] {Calculate \ac{ce}};
        \node (end2) [startstop, below of=pro3, yshift=-.5cm] {\ac{pmv}(\ac{tdb} - \ac{ce}, \ac{tr} - \ac{ce}, \ac{rh}, \ac{v} = \qty{0.1}{\m\per\s}, \ac{met}, \ac{clo})};

        \draw [arrow] (start) -- (dec1);
        \draw [arrow] (dec1) -- node[above,pos=0.3] {Yes} (pro1a);
        \draw [arrow] (dec1) -- node[right,pos=0.3] {No} (dec2);
        \draw [arrow] (pro1a) -- ($(pro1a.south)+(0,-0.5)$) -| (dec2);
        \draw [arrow] (dec2) -- node[above,pos=0.3] {Yes} (pro2a);
        \draw [arrow] (dec2) -- node[right,pos=0.3] {No} (dec3);
        \draw [arrow] (pro2a) -- ($(pro2a.south)+(0,-0.5)$) -| (dec3);
        \draw [arrow] (dec3) -- ($(dec3.west)$) -| node[above,pos=0.3] {No} (end1);
        \draw [arrow] (dec3) -- ($(dec3.east)$) -| node[above,pos=0.3] {Yes} (pro3);
        \draw [arrow] (pro3) -- (end2);
    \end{tikzpicture}\label{fig:flowchart_pmv_ce}
\end{figure}

This is a source of confusion for many users worldwide since both the \gls{7730} and the \gls{55} are the widely adopted and incorporated in many national and international building codes worldwide.
To further complicate this issue, over the course of the last \num{50} years, several other formulations of the \ac{pmv} model have been proposed.
Among the most notable formulations there are the: \gls{pmvs}, \gls{pmvg}, \gls{epmv}, \gls{apmv}, and \ac{athb}.

% todo pmv SET
% todo pmv Gagge

In 2002, Fanger, P. O. and Toftum, J. developed the \gls{epmv} an extension of the \ac{pmv}.
The \gls{epmv} includes an expectancy factor introduced for use in non-air-conditioned buildings in warm climates.
The authors argued that people adapt to the warm environment by decreasing their \ac{met} rate.
They, consequently, propose to reduce the metabolic rate used to calculate the \ac{pmv} by \qty{6.7}{\percent} for every scale unit of \ac{pmv} above neutral.
The resulting \ac{pmv} value is multiplied by the \ac{ef}.
The \ac{ef} for non-air-conditioned buildings can be determined by the length of warm weather throughout the year and whether such buildings may be compared to many others in the region that are air-conditioned.
If the temperature is hot all year or most of the year and there are no or few other air-conditioned buildings, \ac{ef} may be \num{0.5}, but if there are many other air-conditioned buildings, \ac{ef} may be \num{0.7}.
For non-air-conditioned buildings in locations where the temperature is warm solely during the summer and no or few buildings have air-conditioning \ac{ef} is \numrange{0.7}{0.8}, whereas it is \numrange{0.8}{.9} in areas where there are numerous air-conditioned buildings.
In regions with only brief periods of warm weather during the summer, the expectancy factor may be \numrange{0.9}{1}.
This arbitrary and non-scientific definition of the expectancy factor make the \gls{epmv} difficult to calculate and non-generalizable.
Consequently, the \gls{epmv} has not gained any widespread adoption.

Yao, R. et al. (2009) developed the Adaptive Predicted Mean Vote Model~\gls{apmv} to account for occupants adaptation.
They used a ``Black Box'' model to estimate an adaptive coefficient which is in turn use to calculate the \gls{apmv} using the following equation: $\mathrm{aPMV} = \mathrm{PMV}/(1+\lambda \mathrm{PMV})$
The main issue with this approach is that the adaptive coefficient cannot be known a priori, but it is estimated based on the \ac{tsv} collected by a cohort of participants.
This approach cannot be, therefore, generalizable and similarly to the \gls{epmv} model the \gls{apmv} is rarely used in thermal comfort research.

% todo describe marcel's work

To our knowledge, no other study compared the accuracy all the aforementioned aforementioned PMV models.
\mycite{Cheung2019} determined the accuracy of the \ac{pmv} model, by comparing its results with the \ac{tsv} from the ASHRAE ASHRAE Global Thermal Comfort Database II.
While \textbf{COBEE paper} compared the accuracy of the \gls{pmv-ce} and the \ac{pmv} models.

This study aims to specifically fill this knowledge gap by comparing the results of the \ac{pmv} formulations developed over the course of the last years with self-reported \ac{tsv} contained in the ASHRAE Global Thermal Comfort Database II v2.1, referred to as the \gls{db2} thereafter.


