%! Author = sbbfti
%! Date = 8/06/2022

% Preamble
\documentclass[11pt]{article}

\title{PMV model comparison}
\author{Federico Tartarini, Stefano Schiavon}

\usepackage{mypreamble}

\usepackage[printonlyused]{acronym}

\begin{document}

    \maketitle

    \begin{abstract}
        The \ac{pmv} model developed by Fanger predicts thermal sensation.
        It is used in its original formulation in the ISO 7730:2005, while the ASHRAE 55:2020 uses a modified version \acs{pmv-ce} to better account for air speeds.
        We compared their results against XXX thermal sensation votes collected in buildings.
    \end{abstract}

    \section*{Nomenclature}
    %! Author = federico
%! Date = 10/06/2020

\section*{Nomenclature}
\renewcommand{\baselinestretch}{0.75}\normalsize
\renewcommand{\aclabelfont}[1]{\textsc{\acsfont{#1}}}
\begin{acronym}[longest]

    \acro{tdb}[$t_{db}$]{dry-bulb air temperature\acroextra{, $^{\circ}$C}}
    \acro{twb}[$t_{wb}$]{wet-bulb air temperature\acroextra{, $^{\circ}$C}}
    \acro{ti}[$t_{i}$]{indoor air temperature\acroextra{, $^{\circ}$C}}
    \acro{tout}[$t_{out}$]{outdoor air temperature\acroextra{, $^{\circ}$C}}
    \acro{to}[$t_{o}$]{operative air temperature\acroextra{, $^{\circ}$C}}
    \acro{tcl}[$t_{cl}$]{clothing temperature\acroextra{, $^{\circ}$C}}
    \acro{tg}[$t_{g}$]{globe temperature\acroextra{, $^{\circ}$C}}
    \acro{rh}[$RH$]{relative humidity\acroextra{, \%}}
    \acro{v}[$V$]{average air speed\acroextra{, m/s}}
    \acro{tr}[$\overline{t_{r}}$]{mean radiant temperature\acroextra{, $^{\circ}$C}}
    \acro{clo}[$I_{cl}$]{total clothing insulation\acroextra{, clo}}
    \acro{i-cl}[$i_{cl}$]{permeation efficiency of water vapor through the clothing layer}
    \acro{met}[$M$]{rate of metabolic heat production\acroextra{, W/m\textsuperscript{2}}}

    \acro{athb}[ATHB]{Adaptive Thermal Heat Balance}
    \acro{pmv}[PMV]{Predicted Mean Vote}
    \acro{ppd}[PPD]{Predicted Percentage of Dissatisfied\acroextra{, \%}}
    \acro{set}[SET]{Standard Effective Temperature\acroextra{, $^{\circ}$C}}
    \acro{ce}[CE]{Cooling Effect\acroextra{, $^{\circ}$C}}
    \acro{tsv}[TSV]{Thermal Sensation Vote}
    \acro{tpv}[TPV]{Thermal Preference Vote}
    \acro{ef}[$e$]{expectancy factor}
    \acro{af}[a]{adaptive coefficient}

        \acro{i-cl}[$i_{cl}$]{permeation efficiency of water vapor through the clothing layer}
    \acro{met}[$M$]{rate of metabolic heat production\acroextra{, W/m\textsuperscript{2}}}
    \acro{work}[$W$]{rate of mechanical work accomplished\acroextra{, W/m\textsuperscript{2}}}
    \acro{t-sk}[$t_{sk}$]{skin mean temperature\acroextra{, $^{\circ}$C}}
    \acro{t-sk-n}[$t_{sk,n}$]{neutral skin mean temperature\acroextra{, $^{\circ}$C}}
    \acro{t-cr}[$t_{cr}$]{core mean temperature\acroextra{, $^{\circ}$C}}
    \acro{t-re}[$t_{re}$]{rectal temperature\acroextra{, $^{\circ}$C}}
    \acro{t-cr-n}[$t_{cr,n}$]{neutral core mean temperature\acroextra{, $^{\circ}$C}}
    \acro{r-cl}[$R_{cl}$]{thermal resistance of clothing\acroextra{, m\textsuperscript{2}K/W}}
    \acro{r-e-cl}[$R_{e,cl}$]{evaporative heat transfer resistance of clothing layer\acroextra{, m\textsuperscript{2}kPa/W}}
    \acro{f-cl}[$f_{cl}$]{clothing area factor $A_{cl}/A_{body}$\acroextra{, m\textsuperscript{2}K/W}}
    \acro{h}[$h$]{sum of convective $h_{c}$ and radiative $h_{r}$ heat transfer coefficients\acroextra{, W/(m\textsuperscript{2}K)}}
    \acro{h-r}[$h_{r}$]{linear radiative heat transfer coefficient\acroextra{, W/(m\textsuperscript{2}K)}}
    \acro{h-c}[$h_{c}$]{convective heat transfer coefficient\acroextra{, W/(m\textsuperscript{2}K)}}
    \acro{h-e}[$h_{e}$]{evaporative heat transfer coefficient\acroextra{, W/(m\textsuperscript{2}kPa)}}
    \acro{a}[$\alpha$]{fraction of the total body mass considered
to be thermally in the skin compartment}
        \acro{s-cr}[$S_{cr}$]{rate of heat storage in the core compartment\acroextra{, W/m\textsuperscript{2}}}
    \acro{s-sk}[$S_{sk}$]{rate of heat storage in the skin compartment\acroextra{, W/m\textsuperscript{2}}}
    \acro{s}[$S$]{rate of heat storage in the human body\acroextra{, W/m\textsuperscript{2}}}
    \acro{e-res}[$E_{res}$]{rate of evaporative heat loss from respiration\acroextra{, W/m\textsuperscript{2}}}
    \acro{e-dif}[$E_{dif}$]{rate of evaporative heat loss from moisture diffused through the skin\acroextra{, W/m\textsuperscript{2}}}
    \acro{e-rsw}[$E_{rsw}$]{rate of evaporative heat loss from sweat evaporation\acroextra{, W/m\textsuperscript{2}}}
    \acro{e-sk}[$E_{sk}$]{total rate of evaporative heat loss from skin\acroextra{, W/m\textsuperscript{2}}}
    \acro{e-max}[$E_{max}$]{maximum rate of evaporative heat loss from skin\acroextra{, W/m\textsuperscript{2}}}
    \acro{c-res}[$C_{res}$]{rate of convective heat loss from respiration\acroextra{, W/m\textsuperscript{2}}}
    \acro{c-r}[$C + R$]{sensible heat loss from skin\acroextra{, W/m\textsuperscript{2}}}
    \acro{q-res}[$q_{res}$]{total rate of heat loss through respiration\acroextra{, W/m\textsuperscript{2}}}
    \acro{q-sk}[$q_{sk}$]{total rate of heat loss from skin\acroextra{, W/m\textsuperscript{2}}}
    \acro{w}[$w$]{skin wettedness}
    \acro{w-max}[$w_{max}$]{skin wettedness practical upper limit}
    \acro{m-sweat}[$m_{rsw}$]{rate at which regulatory sweat is generated\acroextra{, mL/h\textsuperscript{2}}}
    \acro{m-bl}[$m_{bl}$]{skin blood flow\acroextra{, L/(hm\textsuperscript{2})}}
    \acro{c-sw}[$c_{sw}$]{driving coefficient for regulatory sweating\acroextra{, g/(hKm\textsuperscript{2})}}

    \acro{7730}[ISO~7730:2005]{ISO~7730:2005 is a thermal comfort standard developed by ISO}
    \acro{55}[ASHRAE~55:2023]{ASHRAE~55:2023 is a thermal comfort standard developed by ANSI and ASHRAE}
    \acro{pmv-ce}[PMV\textsubscript{CE}]{PMV calculated in accordance with the ASHRAE~55:2023}
    \acro{epmv}[ePMV]{Adjusted Predicted Mean Votes with Expectancy Factor}
    \acro{apmv}[aPMV]{Adaptive Predicted Mean Vote}
    \acro{pmvs}[PMV$_{SET}$]{PMV calculated using the SET temperature in place of $t_{db}$}
    \acro{pmvg}[PMV$_{Gagge}$]{PMV calculated using the two-node heat balance model}
    \acro{db2}[DB$_{comfort}$]{ASHRAE Global Thermal Comfort Database II}
    \acro{lowess}[LOWESS]{LOcally WEighted Scatterplot Smoothing}

\end{acronym}
\renewcommand{\baselinestretch}{1}\normalsize


    \section{Introduction}\label{sec:introduction}
    In 1970, Fanger~\cite{Fanger1970} developed the \ac{pmv} model, which is now incorporated into the ISO 7730:2005 Standard~\cite{iso7730} in its original form.
    The \ac{pmv} is the most extensively used thermal comfort index for estimating the thermal sensation of a group of people sharing the same thermal environment.
    It is used both by researchers and practitioners.
    However, Cheung et al. (2019) demonstrated that the accuracy of the \ac{pmv} in predicting self-reported thermal sensation of people is only \qty{34}{\percent}~\cite{Cheung2019}.

    ASHRAE-55 2020 Standard utilizes a modified version of the PMV model [4] that calculates a Cooling Effect (CE) using the Standard Effective Temperature (SET) equation for air speeds higher than \qty{0.1}{\m\per\s}.
    The value of CE is then subtracted from both the measured dry-bulb air temperature (tdb) and \ac{tr} and these results become the input into the PMV model.
    Since the CE already accounts for convective heat losses from the person to the environment, the air speed used to calculate the PMV value is set to 0.1 m/s.

    \printbibliography

\end{document}