%! Author = sbbfti
%! Date = 26/08/2021

\section{Conclusions}\label{sec:conclusions}
The present study aimed to compare the accuracy of the \ac{pmv} model, as used in the \gls{7730} Standard, and its modified version, the \ac{pmv-ce} model, as employed in the \gls{55} Standard.
Our findings revealed notable disparities in the predictive accuracy of the \ac{pmv} and \ac{pmv-ce} models when compared against the \ac{tsv} available in the ASHRAE Global Thermal Comfort Database II v2.1 (Comfort DB).
These disparities shed light on the limitations and strengths of the \ac{pmv} and \ac{pmv-ce} models, enabling a more informed decision-making process for the selection of an appropriate formulation within National and International Standards.
The outcomes of this analysis have significant implications for researchers, practitioners, and policymakers involved in the field of thermal comfort assessment and standardization.
% todo add also as a result the fact that thermal sensation -1 to 1 does not mean comfortable
% todo report also this result: Both \ac{pmv} formulations underestimated the conditions at which participants are thermally dissatisfied with their environment. The overall accuracy of the \ac{pmv} and \ac{pmv-ce} were \qty{33}{\percent} and \qty{34}{\percent}, respectively.