%! Author = sbbfti
%! Date = 10/06/2020

% Document

\section{Introduction}\label{sec:introduction}

    In 1970, Fanger~\cite{Fanger1970} developed the \ac{pmv} model, which is now incorporated into the ISO 7730:2005 Standard~\cite{iso7730} in its original form.
    The \ac{pmv} is the most extensively used thermal comfort index for estimating the thermal sensation of a group of people sharing the same thermal environment.
    It is used both by researchers and practitioners.
    However, Cheung et al. (2019) demonstrated that the accuracy of the \ac{pmv} in predicting self-reported thermal sensation of people is only \qty{34}{\percent}~\cite{Cheung2019}.

    ASHRAE 55:2020 Standard utilizes a modified version of the PMV model [4] that calculates a Cooling Effect (CE) using the Standard Effective Temperature (SET) equation for air speeds higher than \qty{0.1}{\m\per\s}.
    The value of CE is then subtracted from both the measured dry-bulb air temperature (tdb) and \ac{tr} and these results become the input into the PMV model.
    Since the CE already accounts for convective heat losses from the person to the environment, the air speed used to calculate the PMV value is set to 0.1 m/s.


