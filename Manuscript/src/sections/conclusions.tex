\section{Conclusions}\label{sec:conclusions}
The present study compares the accuracy of the \ac{pmv} model, as used in the \gls{7730} standard, and its modified version, the \ac{pmv-ce} model, included in the \gls{55} standard.
The \ac{pmv-ce} was intentionally developed to more accurately estimate the effect of elevated air movement~\cite{arens_moving_2009}.
We determined the predictive accuracy of the \ac{pmv} and \ac{pmv-ce} models by comparing their results to the \ac{tsv} available in the \acl{db2}.

% todo check values manually
The overall accuracy and F1-macro score of the \ac{pmv} is \var{overall_acc_pmv} and \num{.16}, respectively.
While the overall accuracy and F1-macro score of the \ac{pmv-ce} is \var{overall_acc_pmv_ce} and \num{.15}, respectively.
% todo check values manually
Both models have an accuracy lower or equal to \qty{5}{\percent} when predicting either `warm', `hot', or `cold'.
This value is lower than random guessing (\qty{14}{\percent}).
This means that the models are misleading the users.
% todo check values manually
When only comparing the results of the two models for entries with \ac{vr}~$\geq$~\qty{0.2}{\m\per\s} the \ac{pmv} outperformed the \ac{pmv-ce} on different metrics.
For example, the median bias of the \ac{pmv} model is \var{bias_median_pmv_0.2} while the median bias for the \ac{pmv-ce} is \var{bias_median_pmv_ce_0.2}.

If we assume that selecting `warmer' or `cooler' on the thermal preference scale indicates thermal discomfort, then our results suggest that individuals who report being `slightly warm' or `slightly cool’ on the thermal sensation scale are thermally comfortable is incorrect. 
This misinterpretation can lead to an underestimation of thermal discomfort. 
In the \ac{db2} \qty{68}{\percent} of participants who reported to be `slightly warm' wanted to be `cooler' and \qty{54}{\percent} of participants who reported to be `slightly cool' wanted to be `warmer'.

\subsection{Implications for the \gls{55} and \gls{7730} Standards and \ac{pmv} Users}\label{subsec:implications-for-the-ashrae-55-and-iso-7730-standards}
Researchers, educators, and practitioners worldwide may wonder which model they should use.
The \gls{55} standard explains how the \ac{pmv-ce} works, however, we could not find a peer-reviewed publication that proves it has superior prediction performances.
The \ac{pmv-ce} is more complex and computationally intensive to run, while not being more accurate than the original \ac{pmv} model, moreover, it does not explicitly solve the lack of proper sweat rate modeling of the \ac{pmv} model. 
Hence, we suggest researchers should calculate the \ac{pmv} values according to the \gls{7730} standards.

The low prediction accuracy of both model formulations, along with the lack of data at the extremes of their applicability limits, suggests that these limits should be reconsidered.
Based on our results, we recommend restricting the use of the \ac{pmv} only to values $\lvert \textrm{PMV}\lvert \leq 0.5$.
In other words, we only recommend using it to determine whether an environment may be deemed thermally neutral by a large group of occupants.
The model, however, is not capable of determining the degree by which people are dissatisfied and can only be used to classify if individuals are on the `cold' (\ac{pmv}~$\leq 0.5$) or `warm' (\ac{pmv}~$\geq 0.5$) side.
The \gls{55} standard specifies only the conditions that are considered comfortable (within the 0.5 range) but does not provide any explicit interpretation for values above or below that range, other than stating that they are not compliant with the standard.
Several peer-reviewed papers \cite{Cheung2019, Yao2022, kim2019thermal, tartarini2018thermal, Humphreys2002, doherty_evaluation_1988} provided corroborating evidence that supports our results.
We understand that this may raise the concern that limiting the \ac{pmv} to $\lvert \textrm{PMV}\lvert \leq 0.5$ would increase the building energy consumption. 
However, this is not necessarily the case.
As shown in Figure~\ref{fig:comfort_regios_pmv_pmvce} increasing \ac{vr} to a modest \qty{0.4}{\m\per\s}, which is achievable with any standard electric fan, is sufficient in an office setting to extend the upper boundary of comfort region to \ac{tdb}~=~\qty{27.7}{\celsius}.
This is the upper \ac{tdb} value estimated by the \ac{pmv} to keep a person within the comfort region while wearing a typical office attire \ac{clor}~=~\qty{0.5}{clo} and performing office tasks \ac{met}~=~\qty{1.2}{met} with \ac{rh}~=~\qty{50}{\percent}.
Hence, we would like to point out that it is not the \ac{pmv} model that is a limiting factor in saving energy but instead is that we cool the air rather than move it.
A field study in Singapore showed that allowing participants to control \ac{v} while simultaneously increasing \ac{tdb} from \qty{24}{\celsius} to \qty{26.5}{\celsius} reduced the energy consumption by \qty{32}{\percent}~\cite{kent_energy_2023}.

We also recommend restricting the ranges on both environmental and personal factors set by both \gls{55} and \gls{7730} standards.
Currently, these limits extend far beyond the range of data contained in the \ac{db2} dataset.
Being the \ac{db2} the largest thermal comfort dataset in the world and containing data from all continents, we deem it not to be necessary to have such wide applicability limits for both \gls{55} and \gls{7730} standards when these environmental variables are rarely observed in buildings across the world.
The lack of data available beyond the ranges depicted in Figure~\ref{fig:dist_input_data} did not allow us to test the accuracy of both the \ac{pmv} and \ac{pmv-ce} above those ranges.
Consequently, we suggest reducing both standards' applicability limits to the ranges shown in Table~\ref{tab:ranges}, until more data are collected.
\begin{table}[htb!]
    \centering
    \begin{tabular}{cc}
        \toprule
        Variable & New Proposed range \\
        \midrule
        \ac{tdb} & \qtyrange{17}{30}{\celsius} \\
        \ac{tr} & \qtyrange{17}{30}{\celsius} \\
        \ac{rh} & \qtyrange{10}{80}{\percent} \\
        \ac{clo} & \qtyrange{0.3}{1.5}{clo} \\
        \ac{met} & \qtyrange{1}{2}{met} \\
        \ac{pmv} & \numrange{-.5}{.5} \\
        \bottomrule
    \end{tabular}
    \caption{New proposed applicability limits for the \gls{55} and \gls{7730} standards.}
    \label{tab:ranges}
\end{table}
\mycite{Humphreys2002} reached a similar conclusion and observed that the ranges of inputs consistent with a valid use of PMV are much narrower than those given in \gls{7730}.
The limit on \ac{met} is also corroborated by evidence provided by \mycite{doherty_evaluation_1988}.
We then recommend that both the \gls{55} and \gls{7730} should change back the upper value of \ac{met} to \qty{2}{met} as it was in the 2020 version of the standard.

The \ac{tsv} is ambiguous and does not allow determining if an action should be taken to increase the participant's thermal comfort.
Assuming that people who are `slightly warm' or `slightly cool' are comfortable leads to undercounting thermal discomfort in our dataset, which is not a valid assumption.
We recommend the use of the thermal preference scale to determine how people perceive their thermal environment. 
This scale more clearly allows an HVAC system to decide whether to change or not the thermal environment.

Finally, we would like to highlight that the current definition of the \ac{pmv} model which states that the model aims ``\ldots to predict the average thermal sensation of a large group of occupants'' is ambiguous.
Both the \gls{55} and \gls{7730} standards should clarify the minimum number of occupants required to use the \ac{pmv} model.
For example, can it be used for a space with only one occupant?

\section*{Acknowledgements}\label{sec:acknowledgements}
We would like to thank Prof. Edward Arens, Dr. Charlie Huizenga and Dr. Paul Raftery for their comments and suggestions on the manuscript.