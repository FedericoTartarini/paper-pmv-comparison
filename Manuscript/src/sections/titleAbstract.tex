\begin{frontmatter}

    \title{PMV: Comparing the prediction accuracy of ISO 7730:2005 and ASHRAE 55:2023}
    % alternatives PMV: Which formulation should you use the ISO 7730:2005 or ASHRAE 55:2020?
    % alternatives ISO 7730:2005 vs ASHRAE 55:2023: A comparison on the accuracy of the two implementations of the PMV model

    \author[label1,label2]{Federico Tartarini\corref{mycorrespondingauthor}}
    \ead{federicotartarini@berkeley.edu}
    \author[label3]{Stefano Schiavon}

    \address[label1]{Berkeley Education Alliance for Research in Singapore, Singapore}
    \address[label2]{Heat and Health Research Incubator, Faculty of Health and Medicine, University of Sydney, Sydney, AU}
    \address[label3]{Center for the Built Environment, University of California, Berkeley, CA, USA}

    \cortext[mycorrespondingauthor]{Corresponding author}

    \begin{abstract}
        The \ac{pmv} model was developed by Fanger more than 50 years ago to predict the mean thermal sensation of a group of people. 
        It is based on a heat balance model of the human body.

        \gls{55} states that the model can be used to determine ``the combinations of indoor thermal environmental factors and personal factors that will produce satisfactory thermal environmental conditions for a majority of the occupants within the space''.

        The \ac{pmv} has seen widespread adoption and has been used by professionals to design and operate buildings and by researchers to predict thermal sensation.
        It is still currently used in its original form in the \gls{7730} although it has a low prediction accuracy, especially outside thermal neutrality.
        Several other formulations have been proposed to compensate for its limitations, one of which is currently used in the \gls{55}.
        \gls{7730} and \gls{55} are the most widely used Standards worldwide, and their differences are a source of confusion for users who find it difficult to decide which model to use.

        We aim to solve this problem by comparing the accuracy of the two main \ac{pmv} formulations, against \var{entries_db_used} thermal sensation votes collected in buildings.
        We found that the \ac{pmv-ce} has the same accuracy as the original \ac{pmv} in predicting thermal neutrality, despite its increased complexity and computational resources needed to calculate the results.
        Surprisingly, the bias of the \ac{pmv-ce} model was greater (a higher number is negative) than the \ac{pmv} when we only analyzed the entries with \ac{v}$\geq$\qty{0.1}{\m\per\sec}.

        % todo add values below and complete the Abstract
        The \ac{pmv} and \ac{pmv-ce} accuracy were XX\% and YY\% for $\mid$\ac{pmv}$\mid \leq 2$ and XX\% and YY\% for $\mid$\ac{pmv}$\mid \leq 1$, respectively.
        Neither PMV model could accurately predict `cold’, `cool’, `warm’, or `hot’ thermal sensations.
        Consequently, the PMV should only be used in conditions close to neutrality within the range $\mid$\ac{pmv}$\mid \leq 1$, and we suggest reducing the applicability limit of the Standards to this range.
    \end{abstract}

    \begin{keyword}
        Thermal comfort model \sep ISO 7730 \sep ASHRAE 55 \sep Thermal sensation \sep Accuracy
    \end{keyword}

\end{frontmatter}
