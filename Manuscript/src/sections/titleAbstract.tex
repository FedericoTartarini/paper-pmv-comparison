%! Author = sbbfti
%! Date = 10/06/2020

\begin{frontmatter}

    \title{PMV formulations: which one is the most accurate?}

    \author[label1]{Federico Tartarini\corref{mycorrespondingauthor}}
    \ead{federicotartarini@berkeley.edu}
    \author[label2]{Stefano Schiavon}

    \address[label1]{Berkeley Education Alliance for Research in Singapore, Singapore}
    \address[label2]{Center for the Built Environment, University of California, Berkeley, CA, USA}

    \cortext[mycorrespondingauthor]{Corresponding author}

    \begin{abstract}
        The \ac{pmv} model was originally developed by Fanger to predict the thermal sensation of a large group of occupants sharing the same thermal environment.
        His model has seen widespread adoption and it is currently used in the \gls{7730}.
        However, to compensate for the model low prediction accuracy several other formulations have been proposed, one of which is currently used in the \gls{55}.
        This is a source of confusion for many practitioners and researchers who find it difficult to settle on which formulation they should use.
        We aim to solve this problem by comparing the accuracy of the main \ac{pmv} formulations against \var{tot_surveys} thermal sensation votes collected in buildings.
        % We found that for V>0.1m/s the \gls{pmv-ce} is significantly more accurate (48%) than the PMV (42%) in predicting thermal neutrality.
        % In 95% of the cases, the error was less than one point.
        % The PMV and PMVCE accuracy were 35% and 36% for |PMV|≤2 and 43% and 44% for |PMV|≤1, respectively.
        % Neither PMV model could accurately predict ‘cold’, ‘cool’, ‘warm’, or ‘hot’ thermal sensations.
        % Consequently, the PMV should only be used in conditions close to neutrality within the range |PMV|≤1.
        % We open-sourced the source code.
    \end{abstract}

    \begin{keyword}
        Personal thermal comfort model \sep Ecological momentary assessment \sep Skin temperature \sep Machine learning \sep Internet of Things (IoT)
    \end{keyword}

\end{frontmatter}
